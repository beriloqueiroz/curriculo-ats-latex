\documentclass[11pt,a4paper,sans]{moderncv}
\moderncvstyle{banking}
\moderncvcolor{black}
\nopagenumbers{}
\usepackage{ifthen}
\usepackage{ifpdf}
\usepackage{color}
\usepackage{lmodern}
\usepackage{marvosym}
\usepackage{url}
\usepackage{longtable}
\usepackage{graphicx}
\usepackage{fancyhdr}
\usepackage[utf8]{inputenc}
\usepackage{ragged2e}
\usepackage[scale=0.915]{geometry}
\usepackage{import}
\usepackage{multicol}
\usepackage{import}
\usepackage{enumitem}
\usepackage[utf8]{inputenc}
\usepackage{amssymb}
\usepackage{umoline}

\name{Luiz Berilo}{Casemiro Queiroz}
\newcommand*{\customcventry}[7][.13em]{
\begin{tabular}{@{}l}
{\bfseries #4} \
{\itshape #3}
\end{tabular}
\hfill
\begin{tabular}{l@{}}
{\bfseries #5} \
{\itshape #2}
\end{tabular}
\ifx&#7&%
\else{\
\begin{minipage}{\maincolumnwidth}%
\small#7%
\end{minipage}}\fi%
\par\addvspace{#1}}
\begin{document}
\makecvtitle
\vspace*{-16mm}
\begin{center}\textbf{ Backend Software Engineer}\end{center}

\section{Contatos}
 {
  \begin{samepage}
	  \faMobile\enspace Telefone: {\color{blue}\href{tel:+5585989071945}{+55 85 9 89071945}} \\
	  \faEnvelope\enspace E-mail: {\color{blue}\href{mailto:berilo.queiroz@gmail.com}{berilo.queiroz@gmail.com}} \\
	  \faHome\enspace Localização: Fortaleza, BR \\
	  \faLinkedin\enspace Linkedin: {\color{blue}\href{https://www.linkedin.com/in/beriloqueiroz}{in/beriloqueiroz}}\\
	  \faGithub\enspace Github: {\color{blue} \href{https://github.com/beriloqueiroz}{github.com/beriloqueiroz}}
  \end{samepage}
 }
\section{Perfil}
 {
  Desenvolvedor backend experiente e orientado a resultados, experiência com importantes tecnologias.
  Histórico comprovado de liderança e entrega de projetos de alta complexidade. Hábil
  em projetar e desenvolver aplicações escaláveis e seguras, focado na resolução de problemas para garantir
  a estabilidade e o desempenho dos produtos. Expertise em metodologias ágeis, arquitetura limpa e
  otimização de código, com um forte compromisso com a inovação e a melhoria contínua.
 }

\section{Habilidades Técnicas}
 {
  Linguagens: \fbox{\strut Java Springboot} \fbox{\strut Go\/Golang} \fbox{\strut Javascript} \fbox{\strut NodeJs} \fbox{\strut typescript} \fbox{\strut C\# .NET}
  \fbox{\strut HTML} \fbox{\strut CSS} \\
  \\
  Banco de dados: \fbox{\strut MySQL} \fbox{\strut PostgreSQL} \fbox{\strut MongoDB} \\
  \\
  Tecnologias: \fbox{\strut API Rest} \fbox{\strut Tests} \fbox{\strut gRPC} \fbox{\strut Docker}\\
  \\
  Arquitetura/Metodologias: \fbox{\strut SOLID} \fbox{\strut Clean Architecture} \fbox{\strut DDD} \fbox{\strut Microservices} \fbox{\strut Scrum} \fbox{\strut Kanban}\\
  \\
  Ferramentas/Nuvem: \fbox{\strut Git} \fbox{\strut Kubernetes} \fbox{\strut AWS Ec2 RDS} \fbox{\strut Google Cloud Run}\\
  \\
 }

\section{Habilidades Comportamentais}
 {
  \fbox{\strut Trabalho em equipe}	\fbox{\strut Boa comunicação} \fbox{\strut Pensamento crítico}\\
  \\
  \fbox{\strut Análise de problemas} \fbox{\strut Visão de negócios} \fbox{\strut Comprometimento}
 }

\section{Experiência Profissional}
\vspace{\baselineskip}

\begin{samepage}
	\customcventry{03/2023 - Atual}{{\color{blue}\href{https://www.buson.com/}{Buson}}}%
	{Desenvolvedor FullStack II}{---}{---}{%
		\begin{itemize}[leftmargin=0.6cm, label={\textbullet}]
			\item Desenvolvimento FullStack utilizando HTML, CSS, Javascript, Nodejs, Typescript, VueJs, Svelte, Handlebars, Java, Spring Boot.
			\item Atuação no squad do e-commerce, operando procedimentos de deploy, ajustando erros de build no Jenkins e Argos, e realizando deploy em produção utilizando Kubernetes.
			\item Participa e contribui ativamente no entendimento da complexidade de novas \textit{features}, bem como nas decisões técnicas.
			\item Contribuiu para entrega de várias \textit{features} que impactam positivamente a receita.
			\item Implementei microsserviço escalável e resiliente, com cobertura de mais de 80\% de testes e com pipeline de CI/CD.
		\end{itemize}
	}
\end{samepage}
\vspace{\baselineskip}

\begin{samepage}
	\customcventry{05/2022 - 02/2023}{{\color{blue}\href{https://www.buson.com/}{Buson}}}%
	{Desenvolvedor FullStack}{---}{---}{%
		\begin{itemize}[leftmargin=0.6cm, label={\textbullet}]
			\item Desenvolvimento FullStack utilizando HTML, CSS, Javascript, Nodejs, Typescript, VueJs, Svelte (POC), Handlebars, Java, Spring Boot.
			\item Atuação no squad do PDV e e-commerce.
			\item Contribuição em projetos focados na experiência do usuário (B2B e B2C), entregando mais de 10 features que melhoraram significativamente a usabilidade.
			\item Contribuição para melhora da taxa de conversão do e-commerce em aproximadamente 15\%.
		\end{itemize}
	}
\end{samepage}
\vspace{\baselineskip}

\begin{samepage}
	\customcventry{01/2021 - 04/2022}{{\color{blue}\href{https://www.maquiadoro.com/}{Maquiadoro}}}%
	{Desenvolvedor FullStack e Diretor}{---}{---}{%
		\begin{itemize}[leftmargin=0.6cm, label={\textbullet}]
			\item Desenvolvimento Backend em NodeJs, Typescript, com foco em SOLID e clean architecture.
			\item Utilização de Docker, MongoDB, Postgres e MySQL.
			\item Frontend em ReactJs em alguns projetos. Manutenção no front-end do e-commerce.
			\item Consultoria de tecnologia e direcionamento da equipe de TI, aplicando metodologia ágil SCRUM.
			\item Contribuiu para maior independência tecnológica na integração ERP e plataforma de e-commerce, desenvolvendo um hub de integração.
			\item Aumentei a eficiência da equipe em mais de 20\% através da implementação de rotinas automatizadas de backoffice.
		\end{itemize}
	}
\end{samepage}
\vspace{\baselineskip}

\begin{samepage}
	\customcventry{05/2019 - 12/2020}{{\color{blue}\href{https://www.maquiadoro.com/}{Maquiadoro}}}%
	{PO/Desenvolvedor e Diretor}{---}{---}{%
		\begin{itemize}[leftmargin=0.6cm, label={\textbullet}]
			\item Atuação como Product Owner e desenvolvedor na equipe de TI, aplicando metodologia ágil SCRUM.
			\item Desenvolvimento de microssistemas para gestão e automação de processos.
			\item Consultoria de tecnologia e direcionamento da equipe de TI.
			\item Contribuiu para maior independência tecnológica na gestão financeira, auxiliando na construção de um sistema de gestão financeira.
			\item Fomentou o entendimento tributário e fiscal com treinamentos constantes.
		\end{itemize}
	}
\end{samepage}
\vspace{\baselineskip}

\begin{samepage}
	\customcventry{08/2018 - 01/2019}{{\color{blue}\href{https://www.maquiadoro.com/}{Maquiadoro}}}%
	{Diretor de e-commerce e operações}{---}{---}{%
		\begin{itemize}[leftmargin=0.6cm, label={\textbullet}]
			\item Responsável pela análise de dados e indicadores, estipulando metas com base no planejamento estratégico.
			\item Mapeamento de processos, análise fiscal e tributária e acompanhamento dos indicadores de negócio.
			\item Contribuição no que se refere a idealização, projeto de 1 loja conceito e em consonância com a omnicanalidade.
			\item Expansão da capacidade operacional da empresa em mais de 20\%, Participação ativa na definição estratégica e definição de metas, propiciando uma gestão mais analítica.
		\end{itemize}
	}
\end{samepage}
\vspace{\baselineskip}

\begin{samepage}
	\customcventry{07/2016 - 07/2018}{{\color{blue}\href{https://www.maquiadoro.com/}{Maquiadoro}}}%
	{Diretor Comercial e de Expedição/logística}{---}{---}{%
		\begin{itemize}[leftmargin=0.6cm, label={\textbullet}]
			\item Expansão logística e gestão do setor comercial.
			\item Análise de vendas, gestão da equipe de compras, análise fiscal e tributária.
			\item Automatização de processos e modernização da empresa.
			\item Reduziu os custos operacionais em mais de 30\% através da otimização de processos logísticos e melhora da relação com stakeholders
		\end{itemize}
	}
\end{samepage}
\vspace{\baselineskip}

\begin{samepage}
	\customcventry{09/2015 - 06/2016}{{\color{blue}\href{https://www.nexoengenharia.com/}{Nexo Engenharia}}}%
	{Engenheiro de Mecatrônica}{---}{---}{%
		\begin{itemize}[leftmargin=0.6cm, label={\textbullet}]
			\item Elaboração e execução de projetos de automação predial e industrial.
			\item Desenvolvimento de programas para controlador lógico programável (CLP) e sistemas de automação.
			\item Configuração de software de supervisão.
			\item Contribuição em obras: Participação do comissionamento, programação de sistemas de contagem de pessoas, medição de água, medição de energia, SDAI e climatização predial de grande porte.
		\end{itemize}
	}
\end{samepage}
\vspace{\baselineskip}

\begin{samepage}
	\customcventry{01/2015 - 09/2015}{{\color{blue}\href{https://www.nexoengenharia.com/}{Nexo Engenharia}}}%
	{Analista de Sistemas de Automação}{---}{---}{%
		\begin{itemize}[leftmargin=0.6cm, label={\textbullet}]
			\item Desenvolvimento de programas para controlador lógico programável (CLP) e sistemas de automação.
			\item Configuração de software de supervisão.
			\item Participação em projetos de automação predial e industrial.
			\item Contribuição em projetos: Realização de mais de 10 orçamentos de automação e SDAI. Participação do comissionamento, programação e stat-up de mais de 1 sistema de automação de climatização de grande porte.
		\end{itemize}
	}
\end{samepage}
\vspace{\baselineskip}

\begin{samepage}
	\customcventry{06/2014 - 01/2015}{{\color{blue}\href{https://www.nexoengenharia.com/}{Nexo Engenharia}}}%
	{Auxiliar de Automação}{---}{---}{%
		\begin{itemize}[leftmargin=0.6cm, label={\textbullet}]
			\item Desenvolvimento de programas para controlador lógico programável (CLP) e sistemas de automação.
			\item Configuração de software de supervisão.
			\item Start-up de sistemas de automação.
			\item Contribuição em projetos: Participou ativamente mais de 4 projetos de automação e SDAI, contribuindo para a implementação bem-sucedida e o start-up dos sistemas.
		\end{itemize}
	}
\end{samepage}
\vspace{\baselineskip}

\begin{samepage}
	\customcventry{12/2012 - 06/2014}{{\color{blue}\href{https://www.nexoengenharia.com/}{Nexo Engenharia}}}%
	{Estagiário}{---}{---}{%
		\begin{itemize}[leftmargin=0.6cm, label={\textbullet}]
			\item Desenvolvimento de programas para controlador lógico programável (CLP) e sistemas de automação.
			\item Configuração de software de supervisão.
			\item Levantamento de quantitativos e pontos de projetos de automação.
			\item Aprendizado e contribuição: Adquiriu conhecimento prático em automação e contribuiu para a entrega de projetos e obras.
		\end{itemize}
	}
\end{samepage}
\vspace{\baselineskip}

\begin{samepage}
	\customcventry{2012 - 2014}{{\color{blue}\href{https://www.ifce.edu.br/}{Laboratório de Processamento de Energia - LPE (IFCE)}}}%
	{Bolsista}{---}{---}{%
		\begin{itemize}[leftmargin=0.6cm, label={\textbullet}]
			\item Desenvolvimento de projetos no âmbito da eletrônica de potência, controle e Energia eólica.
			\item Pesquisa e desenvolvimento: Contribuiu para 2 projetos de pesquisa, resultando em publicações acadêmicas.
		\end{itemize}
	}
\end{samepage}
\vspace{\baselineskip}

\begin{samepage}
	\customcventry{11/2011 - 05/2012}{{\color{blue}\href{https://www.tsautomacao.com.br/}{TS Tecnologia e soluções em automação industrial LTDA}}}%
	{Estagiário}{---}{---}{%
		\begin{itemize}[leftmargin=0.6cm, label={\textbullet}]
			\item Análise de sistemas de automação industrial.
			\item Programação de controladores lógicos programáveis CLP’s.
			\item Implementação eficiente: Participou na implementação de 2 sistemas de automação industrial, melhorando a eficiência dos processos produtivos na área de petróleo e gás.
		\end{itemize}
	}
\end{samepage}
\vspace{\baselineskip}

\begin{samepage}
	\section{Educação}
	\customcventry{2023-2024}{{\color{blue}\href{https://fctech.edu.br}{Faculdade FullCycle de Tecnologia}}}{Pós-graduação (Linguagem Go)}{---}{}{}
	\customcventry{2020-2021}{{\color{blue}\href{https://www.ipog.edu.br}{IPOG}}}{Pós-graduação - MBA em Gestão de Projetos e Processos}{---}{}{}
	\customcventry{2008-2015}{{\color{blue}\href{https://ifce.edu.br}{IFCE, Brasil}}}{Graduação - Engenharia de Mecatrônica}{---}{}{}
\end{samepage}
\vspace{\baselineskip}

\begin{samepage}
	\section{Idiomas}
	\begin{multicols}{2}
		\begin{itemize}[label=\textbullet]
			\item \textbf{English} [Intermediário]
			\item \textbf{Portuguese} [Nativo]
		\end{itemize}
	\end{multicols}
\end{samepage}
\vspace{\baselineskip}

\begin{samepage}
	\section{Cursos \& Certificações}
	 {\begin{itemize}[itemsep=0cm, label=\textbullet]
		  \item Módulo de clean architecture (2023) - \underline{\color{blue}\href{https://fullcycle.com}{FullCycle}}
		  \item Módulo de arquitetura Hexagonal (2023) - \underline{\color{blue}\href{https://fullcycle.com}{FullCycle}}
		  \item Segurança em aplicações WEB (2023) - \underline{\color{blue}\href{https://udemy.com}{Udemy}}
		  \item Módulo de Event Storming (2022) - \underline{\color{blue}\href{https://fullcycle.com}{FullCycle}}
		  \item Módulo de Comunicação entre sistemas (2022) - \underline{\color{blue}\href{https://fullcycle.com}{FullCycle}}
		  \item Módulo de DDD (2022) - \underline{\color{blue}\href{https://fullcycle.com}{FullCycle}}
		  \item Módulo de SOLID Express (2022) - \underline{\color{blue}\href{https://fullcycle.com}{FullCycle}}
		  \item Módulo de Fundamentos de arquitetura de software (2022) - \underline{\color{blue}\href{https://fullcycle.com}{FullCycle}}
		  \item Módulo de Docker (2022) - \underline{\color{blue}\href{https://fullcycle.com}{FullCycle}}
		  \item HTML, CSS, Javascript (2021-2022) - \underline{\color{blue}\href{https://rocketseat.com.br}{Rocketseat}}
		  \item Nodejs, typescript, API REST (express) (2021-2022) - \underline{\color{blue}\href{https://rocketseat.com.br}{Rocketseat - ignite}}
		  \item React (2021-2022) - \underline{\color{blue}\href{https://rocketseat.com.br}{Rocketseat - ignite}}
		  \item HTML - formação básica (2020-2021) - \underline{\color{blue}\href{https://linkedin.com/learning}{LinkedIn Learning}}
		  \item Gestão de orçamentos em projetos (2020-2021) - \underline{\color{blue}\href{https://linkedin.com/learning}{LinkedIn Learning}}
		  \item Gerenciamento de Cronogramas em projetos (2020-2021) - \underline{\color{blue}\href{https://linkedin.com/learning}{LinkedIn Learning}}
		  \item Indústria 4.0: Fundamentos da quarta revolução industrial (2020-2021) - \underline{\color{blue}\href{https://linkedin.com/learning}{LinkedIn Learning}}
		  \item Fundamentos da Inteligência Artificial: Aprendizado de Máquina (2020-2021) - \underline{\color{blue}\href{https://linkedin.com/learning}{LinkedIn Learning}}
		  \item Desenvolvimento de software remoto (2020-2021) - \underline{\color{blue}\href{https://linkedin.com/learning}{LinkedIn Learning}}
		  \item III Programa de formação de consultores (2020-2021) - \underline{\color{blue}\href{https://domani.com.br}{DOMANI}}
		  \item Elipse E3 Desenvolvedores (2015) - \underline{\color{blue}\href{https://elipse.com}{Elipse Software}}
		  \item Curso de curta duração em configurações e desenvolvimento de projetos de CFTV (2015) - \underline{\color{blue}\href{https://axis.com}{Axis Communications}}
		  \item Java SE (2010) - \underline{\color{blue}\href{https://cpqt.com.br}{Centro de Pesquisa e Qualificação Tecnológica}}
	  \end{itemize}}
\end{samepage}

\begin{samepage}
	\section{Outros Projetos}
	\begin{itemize}[itemsep=0cm, leftmargin=0.6cm, label={\textbullet}]
		\item Gerenciador e disparo de notificações web escalável e resiliente. Utilizando Go, DDD, Clean Architecture, Docker, Postgres, Redis, Apache Kafka, Nextjs. {\color{blue}\href{https://github.com/beriloqueiroz/desafio-dev}{github}}
		\item Desenvolvimento de sistema de gestão de pacientes: Utilizando SvelteJs, Go, .NET, Microservices, DDD, Clean Architecture, Docker, AWS - RDS, EC2, VPC, PostgreSQL.
			      {\color{blue}\href{https://github.com/beriloqueiroz/prontu-go}{backend-golang}}, {\color{blue}\href{https://github.com/beriloqueiroz/prontu-front}{frontend-svelte}}, {\color{blue}\href{https://github.com/beriloqueiroz/auth-prontu}{autenticação-backend}}, {\color{blue}\href{https://github.com/beriloqueiroz/prontu}{backend-\.NET}},
		\item Microsserviço genérico de authenticação e autorização usando \.NET C\#, Indentity Server. {\color{blue}\href{https://github.com/beriloqueiroz/auth-user}{github}}
		\item Desenvolvimento do site: Next.js 13 (React.js, ESLint, Tailwind) para {\color{blue}\href{https://snautomacaoeeletrica.com.br}{snautomacaoeeletrica.com.br}}
		\item Desenvolvimento do site: Next.js (React.js, ESLint, Husky, Staged-lint, stylelint, SCSS) para \Midline{bid.log.br}, integrando com roteirizador e rastreamento de encomendas. {\color{blue}\href{https://github.com/beriloqueiroz/bid-site}{github}}
		\item Desenvolvimento do site WordPress: {\color{blue}\href{psicologarichellysousa.com.br}{psicologarichellysousa.com.br}}
	\end{itemize}
\end{samepage}
\vspace{\baselineskip}

\begin{samepage}
	\section{Participações e Palestras}
	\begin{itemize}[itemsep=0cm, leftmargin=0.6cm, label={\textbullet}]
		\item Workshop de Introdução a Microcontroladores e Aplicações - Faculdade Farias Brito (2015)
		\item Workshop de Introdução a Microcontroladores e Aplicações - Universidade de Fortaleza (2014)
		\item Palestrante sobre Automação na V Mostra de Pesquisa em Ciência e Tecnologia (2014)
		\item Minicurso de Introdução à Automação Industrial - Semana da TI na Faculdade Internacional Fanor (2013)
		\item Minicurso de Introdução à Automação Industrial - Semana da Engenharia Elétrica na Faculdade
		\item Internacional Fanor (2013)
	\end{itemize}
\end{samepage}
\vspace{\baselineskip}

\end{document}